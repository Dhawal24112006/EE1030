\let\negmedspace\undefined
\let\negthickspace\undefined
\documentclass[journal]{IEEEtran}
\usepackage[a5paper, margin=10mm, onecolumn]{geometry}
%\usepackage{lmodern} % Ensure lmodern is loaded for pdflatex
\usepackage{tfrupee} % Include tfrupee package
\setlength{\headheight}{1cm} % Set the height of the header box
\setlength{\headsep}{0mm}     % Set the distance between the header box and the top of the text
\usepackage{gvv-book}
\usepackage{gvv}
\usepackage{cite}
\usepackage{amsmath,amssymb,amsfonts,amsthm}
\usepackage{algorithmic}
\usepackage{graphicx}
\usepackage{textcomp}
\usepackage{xcolor}
\usepackage{txfonts}
\usepackage{listings}
\usepackage{enumitem}
\usepackage{mathtools}
\usepackage{gensymb}
\usepackage{comment}
\usepackage[breaklinks=true]{hyperref}
\usepackage{tkz-euclide} 
\usepackage{listings}
% \usepackage{gvv}                                        
\def\inputGnumericTable{}                                 
\usepackage[latin1]{inputenc}                                
\usepackage{color}                                            
\usepackage{array}                                            
\usepackage{longtable}                                       
\usepackage{calc}                                             
\usepackage{multirow}                                         
\usepackage{hhline}                                           
\usepackage{ifthen}                                           
\usepackage{lscape}
\renewcommand{\thefigure}{\theenumi}
\renewcommand{\thetable}{\theenumi}
\setlength{\intextsep}{10pt} % Space between text and floats
\numberwithin{equation}{enumi}
\numberwithin{figure}{enumi}
\renewcommand{\thetable}{\theenumi}
\begin{document}
\bibliographystyle{IEEEtran}
\title{Question 1-1.10-4}
\author{EE24BTECH11015 - Dhawal}
% \maketitle
% \newpage
% \bigskip
{\let\newpage\relax\maketitle}
\begin{enumerate}
\item Find the unit vector in the direction of sum of vectors $\vec{a}$= $2\hat{i}-\hat{j}+\hat{k}$  and  $\vec{b}=2\hat{j}+\hat{k}$.

\end{enumerate}

\begin{table}[h!]    
  \centering
  
\begin{tabular}[12pt]{ |c| c| c|}
    \hline
    \textbf{Variable} & \textbf{Description} & \textbf{Values} \\ 
    \hline
    P & Parabola & $4y = 3x^2$ \\
    \hline
    L & Line & $2y=3x+12$ \\
    \hline
    $\vec{A}$ & Point of intersection & To find \\
    \hline
    $\vec{B}$ & Point of intersection & To find \\
    \hline
   
    \end{tabular}


  \caption{Variables Used}
  \label{tab 1.4.9.2}
\end{table}
Solution:\\

As $\vec{P}$ is a unit vector:
\begin{align}
        \vec{P}=\frac{\vec{a}+\vec{b}}{\mydet{\mydet{\vec{a}+\vec{b}}}}
\end{align}

Finding $\vec{a}+\vec{b}$:
\begin{align}
        \vec{a}+\vec{b}=\myvec{2\\ -1 \\ 1}+ \myvec{0\\2\\1}=\myvec{2\\1\\2}
\end{align}

Finding ${\mydet{\mydet{\vec{a}+\vec{b}}}}$:
\begin{align}
        {\mydet{\mydet{\vec{a}+\vec{b}}}}= \sqrt{\myvec{2&1&2}\myvec{2\\1\\2}}=
        \sqrt{9}=3
\end{align}

Putting the values in the equation:
    \begin{align}
         \vec{P}=\frac{1}{3}\myvec{2\\1\\2}=\myvec{\frac{2}{3}\\\\\frac{1}{3}\\\\\frac{2}{3}}
    \end{align}

Hence unit vector in direction of $\vec{a}+\vec{b}$ is $\frac{2}{3} \hat{i}+ \frac{1}{3}\hat{j}+\frac{2}{3}\hat{k}$
\begin{figure}[h!]
   \centering
   \includegraphics[width=0.9\linewidth]{Figure_1.png}
	\caption{Locus of $\vec{P}$ }
   \label{stemplot}
\end{figure}


\end{document}

