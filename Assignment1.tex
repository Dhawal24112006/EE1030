%iffalse
\let\negmedspace\undefined
\let\negthickspace\undefined
\documentclass[journal,12pt,twocolumn]{IEEEtran}
\usepackage{cite}
\usepackage{amsmath,amssymb,amsfonts,amsthm}
\usepackage{algorithmic}
\usepackage{graphicx}
\usepackage{textcomp}
\usepackage{xcolor}
\usepackage{txfonts}
\usepackage{listings}
\usepackage{enumitem}
\usepackage{mathtools}
\usepackage{gensymb}
\usepackage{comment}
\usepackage[breaklinks=true]{hyperref}
\usepackage{tkz-euclide} 
\usepackage{listings}
\usepackage{multicol}
\usepackage{gvv}                                        
%\def\inputGnumericTable{}                                 
\usepackage[latin1]{inputenc}                                
\usepackage{color}                                            
\usepackage{array}                                            
\usepackage{longtable}                                       
\usepackage{calc}                                             
\usepackage{multirow}                                         
\usepackage{hhline}                                           
\usepackage{ifthen}                                           
\usepackage{lscape}
\usepackage{tabularx}
\usepackage{array}
\usepackage{float}


\newtheorem{theorem}{Theorem}[section]
\newtheorem{problem}{Problem}
\newtheorem{proposition}{Proposition}[section]
\newtheorem{lemma}{Lemma}[section]
\newtheorem{corollary}[theorem]{Corollary}
\newtheorem{example}{Example}[section]
\newtheorem{definition}[problem]{Definition}
\newcommand{\BEQA}{\begin{eqnarray}}
\newcommand{\EEQA}{\end{eqnarray}}

\theoremstyle{remark}


% Marks the beginning of the document
\begin{document}
\bibliographystyle{IEEEtran}
\vspace{3cm}

\title{Chapter 6 Sequence and Series}
\author{ee24btech11015 - Dhawal}
\maketitle
\newpage
\bigskip

\renewcommand{\thefigure}{\theenumi}
\renewcommand{\thetable}{\theenumi}

\section*{D. MCQs with One or More than One Correct}
\begin{enumerate}
    \item If the first and the (2n - 1)th terms of an A.P., a G.P. and an H.P. are equal and their n-th terms are a, b and c respectively, then {(1988-2 Marks)}
\begin{multicols}{2}
\begin{enumerate}
\item $a=b=c$
\item $a \geq b \geq c$
\item $a+b=c$
\item $ac-b^2=0$
\end{enumerate}
\end{multicols}

\item For $0 < \phi < \pi /2$, if 
\begin{equation*}
x=\sum_{n=0}^{\infty} \cos^{2n} \phi ,
y=\sum_{n=0}^{\infty} \sin^{2n} \phi ,
z=\sum_{n=0}^{\infty} \cos^{2n} \phi \sin^{2n} \phi
\end{equation*}
then: \hfill{(1993-2 Marks)}
\begin{multicols}{2}
\begin{enumerate}
\item $xyz=xz+y$
\item $xyz=xy+z$
\item $xyz=x+y+z$
\item $xyz+yz+x$
\end{enumerate}
\end{multicols}

\item Let n be a odd integer. If 
\begin{equation*}
sin n\theta= \sum_{r=0}^{n} b_r sin^{r} \theta, 
\end{equation*}
for every value of $\theta$, then
\hfill{(1998-2 Marks)}
\begin{enumerate}
\item $b_0=1, b_1=3$
\item $b_0=0, b_1=n$
\item $b_0=-1, b_1=3$
\item $b_0=0, b_1=n^2-3n+3$
\end{enumerate}

\item Let $T_r$ be the $r^{th}$ term of an A.P., for r=1,2,3,.... If for some positive integers m,n we have
$T_m=\frac{1}{n}$ and $T_n=\frac{1}{m}$ ,then $T_{mn}$ equals\hfill{(1998-2 Marks)}
\begin{multicols}{2}
\begin{enumerate}
\item $\frac{1}{mn}$
\item $\frac{1}{m} + \frac{1}{m}$
\item $1$
\item $0$
\end{enumerate}
\end{multicols}

\item If $x>1,y>1,z>1$ are in G.P.,then $\frac{1}{1+lnx},\frac{1}{1+lny},\frac{1}{1+lnz}$ are in 
\hfill{(1998-2 Marks)}
\begin{multicols}{2}
\begin{enumerate}
\item A.P.
\item H.P.
\item G.P.
\item None of these
\end{enumerate}
\end{multicols}

\item For a positive integer n, let
$a_n=1+\frac{1}{2}+\frac{1}{3}+\frac{1}{4}+....\frac{1}{(2^n)-1}$.Then \hfill{(1999-2 Marks)}
\begin{multicols}{2}
\begin{enumerate}
\item $a(100)\leq 100$
\item $a(100) > 100$
\item $a(200)\leq 100$
\item $a(200) > 100$
\end{enumerate}
\end{multicols}

\item A straight line through the vertex P of a triangle POR intersects the side QR at the point S and the circumcircle of the triangle PQR at the point T. If S is not the centre of the circumcircle,then  \hfill{(2008)}
\begin{multicols}{2}
\begin{enumerate}
\item $\frac{1}{PS}+\frac{1}{ST}<\frac{2}{\sqrt{QS \cdot SR}}$
\item $\frac{1}{PS}+\frac{1}{ST}>\frac{2}{\sqrt{QS \cdot SR}}$
\item $\frac{1}{PS}+\frac{1}{ST}<\frac{4}{QR}$
\item $\frac{1}{PS}+\frac{1}{ST}>\frac{4}{QR}$
\end{enumerate}
\end{multicols}

\item Let \begin{equation*}S_n=\sum_{k=1}^{n}\frac{n}{n^2+kn+k^2}\;\;and\;\; T_n=\sum_{k=0}^{n-1}\frac{n}{n^2+kn+k^2}\end{equation*}for\;\; n=1,2,3,...... Then,\hfill{(2008)}
\begin{multicols}{2}
\begin{enumerate}
\item $S_n<\frac{\pi}{3\sqrt{3}}$
\item $S_n>\frac{\pi}{3\sqrt{3}}$
\item $T_n<\frac{\pi}{3\sqrt{3}}$
\item $T_n>\frac{\pi}{3\sqrt{3}}$
\end{enumerate}
\end{multicols}

\item Let \begin{equation*} S_n=\sum_{k=1}^{4n}(-1)^\frac{k(k+1)}{2}k^2.\end{equation*}  Then $S_n$ can take value(s)  \hfill{(JEE Adv.2013)}
\begin{multicols}{2}
\begin{enumerate}
\item $1056$
\item $1088$
\item $1120$
\item $1332$
\end{enumerate}
\end{multicols}

\item Let $\alpha$ and $\beta$ be the roots of $x^2-x-1=0$, with $\alpha>\beta$. For all positive integers n, define
\begin{equation*}
a_n=\frac{\alpha_n-\beta_n}{\alpha-\beta},n\geq2
\end{equation*}
\begin{equation*}
b_1=1\:and\:b_n=a_{n-1}+a_{n+1},n\geq1
\end{equation*}
Then which of the following options is/are correct?
\hfill{(JEE Adv. 2019)}
\begin{enumerate}
\item $\sum_{n=1}^{\infty}\frac{a_n}{10^n}=\frac{10}{89}$
\item $B_n=a^n+b^n\:for\: all\: n\geq1$
\item $a_1+a_2+a_3+.....a_n=a_{n+2}-1\: for\: all\: n\geq1$
\item $\sum_{n=1}^{\infty}\frac{b_n}{10^n}=\frac{8}{89}$
\end{enumerate}
\end{enumerate}

\section*{E. Subjective Problems}
\begin{enumerate}
    \item The harmonic mean of two numbers is 4. Their arithmetic mean A and the geometric mean G satisfy the relation.
    \\
    $2A + G^2 = 27$
    \\
    Find the two numbers.  \hfill{(1979)}

\item The interior angles of a polygon are in arithmetic progression. The smallest angle is 120\degree and the common difference is 5\degree Find the number of sides of the polygon. \hfill{(1980)}

    \item Does there exist a geometric progression containing 27, 8 and 12 as three of its terms ? If it exits, how many such progressions are possible?  \hfill{(1982-3 Marks)}

    \item Find three numbers a,b,c, between 2 and 18 such that
    \begin{enumerate}
    \item their sum is 25
    \item the numbers 2,a,b are consecutive terms of an A.P.
    \\
    and
    \item the numbers b,c,18 are consecutive terms of a G.P.\hfill{(1983-2 Marks)}
    \end{enumerate}
  

    \item If $a>0,b>0,c>0$, prove that
    \\
    $(a+b+c)(\frac{1}{a}+\frac{1}{b}+\frac{1}{c})\geq9$
    \hfill{(1984-2 Marks)}

\end{enumerate}


\end{document}
