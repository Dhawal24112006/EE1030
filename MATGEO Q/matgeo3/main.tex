\let\negmedspace\undefined
\let\negthickspace\undefined
\documentclass[journal]{IEEEtran}
\usepackage[a5paper, margin=10mm, onecolumn]{geometry}
%\usepackage{lmodern} % Ensure lmodern is loaded for pdflatex
\usepackage{tfrupee} % Include tfrupee package
\setlength{\headheight}{1cm} % Set the height of the header box
\setlength{\headsep}{0mm}     % Set the distance between the header box and the top of the text
\usepackage{gvv-book}
\usepackage{gvv}
\usepackage{cite}
\usepackage{amsmath,amssymb,amsfonts,amsthm}
\usepackage{algorithmic}
\usepackage{graphicx}
\usepackage{textcomp}
\usepackage{xcolor}
\usepackage{txfonts}
\usepackage{listings}
\usepackage{enumitem}
\usepackage{mathtools}
\usepackage{gensymb}
\usepackage{comment}
\usepackage[breaklinks=true]{hyperref}
\usepackage{tkz-euclide} 
\usepackage{listings}
% \usepackage{gvv}                                        
\def\inputGnumericTable{}                                 
\usepackage[latin1]{inputenc}                                
\usepackage{color}                                            
\usepackage{array}                                            
\usepackage{longtable}                                       
\usepackage{calc}                                             
\usepackage{multirow}                                         
\usepackage{hhline}                                           
\usepackage{ifthen}                                           
\usepackage{lscape}
\renewcommand{\thefigure}{\theenumi}
\renewcommand{\thetable}{\theenumi}
\setlength{\intextsep}{10pt} % Space between text and floats
\numberwithin{equation}{enumi}
\numberwithin{figure}{enumi}
\renewcommand{\thetable}{\theenumi}
\begin{document}
\bibliographystyle{IEEEtran}
\title{Question 1-1.10-9}
\author{EE24BTECH11015 - Dhawal}
% \maketitle
% \newpage
% \bigskip
{\let\newpage\relax\maketitle}
\begin{enumerate}
\item Find the unit vector in the direction of vector $\overrightarrow{PQ}$ ,  where $\vec{P}$ and $\vec{Q}$ are the points
(1,  2,  3) and (4,  5,  6),  respectively.
\end{enumerate}

\begin{table}[h!]    
  \centering
  
\begin{tabular}[12pt]{ |c| c| c|}
    \hline
    \textbf{Variable} & \textbf{Description} & \textbf{Values} \\ 
    \hline
    P & Parabola & $4y = 3x^2$ \\
    \hline
    L & Line & $2y=3x+12$ \\
    \hline
    $\vec{A}$ & Point of intersection & To find \\
    \hline
    $\vec{B}$ & Point of intersection & To find \\
    \hline
   
    \end{tabular}


  \caption{Variables Used}
  \label{tab 1.4.9.2}
\end{table}
Solution:\\

As $\vec{A}$ is a unit vector in the direction of $\overrightarrow{PQ}$ :
\begin{align}
        \vec{A}=\frac{\vec{Q}-\vec{P}}{\mydet{\mydet{\vec{Q}-\vec{P}}}}
\end{align}

Finding $\vec{Q}-\vec{P}$:
\begin{align}
        \vec{Q}-\vec{P}=\myvec{4\\ 5 \\ 6}-\myvec{1\\2\\3}=\myvec{3\\3\\3}
\end{align}

Finding ${\mydet{\mydet{\vec{Q}-\vec{P}}}}$:
\begin{align}
        {\mydet{\mydet{\vec{Q}-\vec{P}}}}= \sqrt{\myvec{3&3&3}\myvec{3\\3\\3}}=
        \sqrt{27}=3\sqrt{3}
\end{align}

Putting the values in the equation:
    \begin{align}
         \vec{A}=\frac{1}{3\sqrt{3}}\myvec{3\\3\\3}=\myvec{\frac{1}{\sqrt{3}}\\\frac{1}{\sqrt{3}}\\\frac{1}{\sqrt{3}}}
    \end{align}

Hence unit vector in direction of $\overrightarrow{PQ}$ is $\myvec{\frac{1}{\sqrt{3}}\\\frac{1}{\sqrt{3}}\\\frac{1}{\sqrt{3}}}$
\begin{figure}[h!]
   \centering
   \includegraphics[width=0.9\linewidth]{Figure_1.png}
	\caption{Vector $\vec{A}$ }
   \label{stemplot}
\end{figure}


\end{document}

