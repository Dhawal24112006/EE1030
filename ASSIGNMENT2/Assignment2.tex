%iffalse
\let\negmedspace\undefined
\let\negthickspace\undefined
\documentclass[journal,12pt,twocolumn]{IEEEtran}
\usepackage{cite}
\usepackage{amsmath,amssymb,amsfonts,amsthm}
\usepackage{algorithmic}
\usepackage{graphicx}
\usepackage{textcomp}
\usepackage{xcolor}
\usepackage{txfonts}
\usepackage{listings}
\usepackage{enumitem}
\usepackage{mathtools}
\usepackage{gensymb}
\usepackage{comment}
\usepackage[breaklinks=true]{hyperref}
\usepackage{tkz-euclide} 
\usepackage{listings}
\usepackage{multicol}
\usepackage{xparse}
\usepackage{gvv}                                        
%\def\inputGnumericTable{}                                 
\usepackage[latin1]{inputenc}                                
\usepackage{color}                                            
\usepackage{array}                                            
\usepackage{longtable}                                       
\usepackage{calc}                                             
\usepackage{multirow}                                         
\usepackage{hhline}                                           
\usepackage{ifthen}                                           
\usepackage{lscape}
\usepackage{tabularx}
\usepackage{array}
\usepackage{float}

\newtheorem{theorem}{Theorem}[section]
\newtheorem{problem}{Problem}
\newtheorem{proposition}{Proposition}[section]
\newtheorem{lemma}{Lemma}[section]
\newtheorem{corollary}[theorem]{Corollary}
\newtheorem{example}{Example}[section]
\newtheorem{definition}[problem]{Definition}
\newcommand{\BEQA}{\begin{eqnarray}}
\newcommand{\EEQA}{\end{eqnarray}}

\theoremstyle{remark}



% Marks the beginning of the document
\begin{document}
\bibliographystyle{IEEEtran}
\vspace{3cm}

\title{Chapter 13 Properties of Triangle}
\author{ee24btech11015 - Dhawal}
\maketitle
\newpage
\bigskip

\renewcommand{\thefigure}{\theenumi}
\renewcommand{\thetable}{\theenumi}

\begin{enumerate}[start =3]
        
    \item The sum of the radii of inscribed and circumscribed circles for an n sided regular polygon of side $a$, is  \hfill{(2003)}
\begin{multicols}{2}
\begin{enumerate}
        \item $\frac{a}{4} \cot{\brak{\frac{\pi}{2n}}}$                    
        \item $ a \cot{\brak{\frac{\pi}{n}}}$ 
        \item $\frac{a}{2} \cot{\brak{\frac{\pi}{2n}}}$ 
        \item $ a \cot{\brak{\frac{\pi}{2n}}}$
\end{enumerate}
\end{multicols}

\item In a triangle ABC, medians AD and BE are drawn. If $AD=4, \angle{DAB}=\frac{\pi}{6}\:and\:\angle{ABE}=\frac{\pi}{3},$ then the area of the $\Delta{ABC}$ is \hfill{(2003)}
\begin{multicols}{4}
\begin{enumerate}
        \item $\frac{64}{3}$                    
        \item $\frac{8}{3}$ 
        \item $\frac{16}{3}$ 
        \item $\frac{32}{3\sqrt{3}}$
\end{enumerate}
\end{multicols} 

\item If in $\Delta{ABC}\;a\cos^2\brak{\frac{C}{2}} + c\cos^2\brak{\frac{A}{2}} = \frac{3b}{2},$ then the sides $a,b\;and\;c$ \hfill{(2003)}
\begin{multicols}{2}
\begin{enumerate}
        \item satisfy $a+b=c$                    
        \item are in A.P. 
        \item are in G.P. 
        \item are in H.P.
\end{enumerate}
\end{multicols} 

\item The sides of a triangle are $\sin\alpha,\cos\alpha\;and\;\sqrt{1+\sin\alpha\cos\alpha}$ for some $0<\alpha<\frac{\pi}{2}.$ Then the greatest angle of the triangle is \hfill{(2004)}
\begin{multicols}{4}
\begin{enumerate}
        \item $150\degree$                    
        \item $90\degree$ 
        \item $120\degree$
        \item $60\degree$
\end{enumerate}
\end{multicols} 

\item A person standing on the bank of a river observes that the angle of elevation of the top of a tree on the opposite bank of the river is $60\degree$ and when he retires $40$ meters away from the tree, the angle of elevation becomes $30\degree$. The breadth of the river is \hfill{(2004)}
\begin{multicols}{4}
\begin{enumerate}
        \item $60m$                    
        \item $30m$ 
        \item $40m$ 
        \item $20m$
\end{enumerate}
\end{multicols} 

\item In a triangle $ABC$, let $\angle{C}=\frac{\pi}{2}.$ If $r$ is the inradius and $R$ is the circumradius of the triangle $ABC,$ then $2\brak{R+r}$ equals \hfill{(205)}
\begin{multicols}{4}
\begin{enumerate}
        \item $b+c$                    
        \item $a+b$ 
        \item $a+b+c$ 
        \item $c+a$
\end{enumerate}
\end{multicols} 

\item If in a $\Delta ABC,$ let the altitudes from the vertices $A,B,C$ on opposite sides are in H.P., then $\sin A,\sin B,\sin C$ are in \hfill{(2005)}
\begin{multicols}{2}
\begin{enumerate}
        \item $G.P.$                    
        \item $A.P.$ 
        \item $A.P.-G.P.$ 
        \item $H.P.$
\end{enumerate}
\end{multicols} 

\item A tower stand at the centre of a circular park. A and B are two points on the boundary of the park such that $AB\brak{=a}$ subtends an angle of $60\degree$ at the foot of the tower, and the angle of elevation of the top of the tower from A or B is $30\degree.$ The height of the tower is \hfill{(2007)}
\begin{multicols}{4}
\begin{enumerate}
        \item $\frac{a}{\sqrt{3}}$                    
        \item $a\sqrt{3}$ 
        \item $\frac{2a}{\sqrt{3}}$ 
        \item $2a\sqrt{3}$
\end{enumerate}
\end{multicols} 

\item AB is a vertical pole with B at the ground level and A at the top. A man finds that the angle of elevation the the point A from a certain point C on the ground is $60\degree.$ He moves away from the pole along the line BC to a point D such that $CD=7m.$ From D the angle of elevation of point A is $45\degree.$ Then the height of the pole is  \hfill{(2008)}
\begin{multicols}{2}
\begin{enumerate}
        \item $\frac{7\sqrt{3}}{2}\frac{1}{\sqrt{3}-1}m$                   
        \item $\frac{7\sqrt{3}}{2}\brak{\sqrt{3}+1}m$ 
        \item $\frac{7\sqrt{3}}{2}\brak{\sqrt{3}-1}m$ 
        \item $\frac{7\sqrt{3}}{2}\frac{1}{\sqrt{3}+1}m$
\end{enumerate}
\end{multicols} 

\item For a regular polygon, let $r$ and $R$ be the radii of the inscribed and the circumscribed circles. A false statement among the following is \hfill{(2010)}
\begin{enumerate}
        \item There is a regular polygon with $\frac{r}{R}=\frac{1}{\sqrt{2}}$                    
        \item There is a regular polygon with $\frac{r}{R}=\frac{2}{3}$ 
        \item There is a regular polygon with $\frac{r}{R}=\frac{\sqrt{3}}{2}$ 
        \item There is a regular polygon with $\frac{r}{R}=\frac{1}{2}$
\end{enumerate} 

\item A bird is sitting on the top of a vertical pole $20m$ high and its elevation from a point O on the  ground is $45\degree.$ It flies off horizontally straight away from the point O. After one second, the elevation of the bird from O is reduced to {30\degree.} Then the speed in \brak{in\:m/s} of the bird is \hfill{(JEE M 2014)}
\begin{multicols}{2}
\begin{enumerate}
        \item $20\sqrt{2}$                    
        \item $20\brak{\sqrt{3}-1}$ 
        \item $40\brak{\sqrt{2}-1}$ 
        \item $40\brak{\sqrt{3}-\sqrt{2}}$
\end{enumerate}
\end{multicols} 

\item If the angle of elevation of the top of a tower from three colinear points $A,B\:and\:C$ on a line leading to foot of the tower, are $30\degree,45\degree \: and \: 60\degree$ respectively, then the ratio, $AB:BC,$ is: 

\hfill{(JEE M 2015)}
\begin{multicols}{2}
\begin{enumerate}
        \item $1:\sqrt{3}$                    
        \item $2:3$ 
        \item $\sqrt{3}:1$ 
        \item $\sqrt{3}:\sqrt{2}$
\end{enumerate}
\end{multicols} 

\item Let a vertical tower AB have its end A on the level ground. Let C be the mid-point of AB and P be a point on the ground such that $AP=2AB.$ If $\angle{BPC}=\beta,$ then $\tan \beta$ is equal to:  

\hfill{(JEE M 2017)}
\begin{multicols}{4}
\begin{enumerate}
        \item $\frac{4}{9}$                    
        \item $\frac{6}{7}$ 
        \item $\frac{1}{4}$ 
        \item $\frac{2}{9}$
\end{enumerate}
\end{multicols} 

\item PQR is a triangular park with $PQ=PR=200m$. A T.V. tower stands at the mid-point of QR. If the angles of the elevation of the top of the tower at P,Q and R are respectively $45\degree , 30\degree \: and \: 30\degree,$ then the height of the tower \brak{in\:m} is: \hfill{(JEE M 2018)}
\begin{multicols}{2}
\begin{enumerate}
        \item $50$                    
        \item $100\sqrt{3}$ 
        \item $50\sqrt{2}$ 
        \item $100$
\end{enumerate}
\end{multicols} 


\end{enumerate}
\end{document}
