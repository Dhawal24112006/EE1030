%iffalse
\documentclass[journal]{IEEEtran}
\usepackage[a5paper, margin=10mm]{geometry}
%\usepackage{lmodern} % Ensure lmodern is loaded for pdflatex
\usepackage{tfrupee} % Include tfrupee package


\setlength{\headheight}{1cm} % Set the height of the header box
\setlength{\headsep}{0mm}     % Set the distance between the header box and the top of the text


%\usepackage[a5paper, top=10mm, bottom=10mm, left=10mm, right=10mm]{geometry}

%
\setlength{\intextsep}{10pt} % Space between text and floats

\makeindex


\usepackage{cite}
\usepackage{amsmath,amssymb,amsfonts,amsthm}
\usepackage{algorithmic}
\usepackage{graphicx}
\usepackage{textcomp}
\usepackage{xcolor}
\usepackage{txfonts}
\usepackage{listings}
\usepackage{enumitem}
\usepackage{mathtools}
\usepackage{gensymb}
\usepackage{comment}
\usepackage[breaklinks=true]{hyperref}
\usepackage{tkz-euclide} 
\usepackage{listings}
\usepackage{multicol}
\usepackage{xparse}
\usepackage{gvv}
%\def\inputGnumericTable{}                                 
\usepackage[latin1]{inputenc}                                
\usepackage{color}                                            
\usepackage{array}                                            
\usepackage{longtable}                                       
\usepackage{calc}                                             
\usepackage{multirow}                                         
\usepackage{hhline}                                           
\usepackage{ifthen}                                           
\usepackage{lscape}
\usepackage{tabularx}
\usepackage{array}
\usepackage{float}


\newtheorem{theorem}{Theorem}[section]
\newtheorem{problem}{Problem}
\newtheorem{proposition}{Proposition}[section]
\newtheorem{lemma}{Lemma}[section]
\newtheorem{corollary}[theorem]{Corollary}
\newtheorem{example}{Example}[section]
\newtheorem{definition}[problem]{Definition}
\newcommand{\BEQA}{\begin{eqnarray}}
\newcommand{\EEQA}{\end{eqnarray}}

\theoremstyle{remark}


\begin{document}
\bibliographystyle{IEEEtran}
\onecolumn

\title{Chapter 13 Properties of Triangle}
\author{ee24btech11015 - Dhawal}
\maketitle

\renewcommand{\thefigure}{\theenumi}
\renewcommand{\thetable}{\theenumi}

\section*{B. JEE Main / AIEEE}
\begin{enumerate}[start=3]
        \item The sum of the radii of inscribed and circumscribed circles for an $n$ sided regular polygon of side $a$, is \hfill\brak{2003}
\begin{multicols}{4}
\begin{enumerate}
        \item $\frac{a}{4} \cot{\brak{\frac{\pi}{2n}}}$         
        \item $ a \cot{\brak{\frac{\pi}{n}}}$ 
        \item $\frac{a}{2} \cot{\brak{\frac{\pi}{2n}}}$ 
        \item $ a \cot{\brak{\frac{\pi}{2n}}}$
\end{enumerate}
\end{multicols}

\item In a triangle $\Delta{ABC}$, medians AD and BE are drawn. If AD$=4,\angle{DAB}=\frac{\pi}{6} \text{ and } \angle{ABE}=\frac{\pi}{3},$ then the area of the $\Delta{ABC}$ is \hfill\brak{2003}
\begin{multicols}{4}
\begin{enumerate}
        \item $\frac{64}{3}$                    
        \item $\frac{8}{3}$ 
        \item $\frac{16}{3}$ 
        \item $\frac{32}{3\sqrt{3}}$
\end{enumerate}
\end{multicols} 

\item If in $\Delta{ABC}\;a\cos^2\brak{\frac{\vec{C}}{2}} + c\cos^2\brak{\frac{\vec{A}}{2}} = \frac{3b}{2},$ then the sides $a,b\text{ and }c$ \hfill\brak{2003}
\begin{multicols}{4}
\begin{enumerate}
        \item satisfy $a+b=c$                    
        \item are in A.P. 
        \item are in G.P. 
        \item are in H.P.
\end{enumerate}
\end{multicols} 

\item The sides of a triangle are $\sin\alpha,\cos\alpha \text{ ad } \sqrt{1+\sin\alpha\cos\alpha}$ for some $0<\alpha<\frac{\pi}{2}.$ Then the greatest angle of the triangle is \hfill\brak{2004}
\begin{multicols}{4}
\begin{enumerate}
        \item $150\degree$                    
        \item $90\degree$ 
        \item $120\degree$
        \item $60\degree$
\end{enumerate}
\end{multicols} 

\item A person standing on the bank of a river observes that the angle of elevation of the top of a tree on the opposite bank of the river is $60\degree$ and when he retires $40$ meters away from the tree, the angle of elevation becomes $30\degree$. The breadth of the river is 

\hfill\brak{2004}
\begin{multicols}{4}
\begin{enumerate}
        \item $60m$                    
        \item $30m$ 
        \item $40m$ 
        \item $20m$
\end{enumerate}
\end{multicols} 

\item In a triangle $\Delta{ABC}$, let $\angle{C}=\frac{\pi}{2}.$ If $r$ is the inradius and $R$ is the circumradius of the triangle $\Delta{ABC},$ then $2\brak{R+r}$ equals \hfill\brak{2005}
\begin{multicols}{4}
\begin{enumerate}
        \item $b+c$                    
        \item $a+b$ 
        \item $a+b+c$ 
        \item $c+a$
\end{enumerate}
\end{multicols} 

\item If in a $\Delta ABC,$ let the altitudes from the vertices $\vec{A,B,C}$ on opposite sides are in H.P., then $\sin \vec{A},\sin \vec{B},\sin \vec{C}$ are in \hfill\brak{2005}
\begin{multicols}{4}
\begin{enumerate}
        \item $G.P.$                    
        \item $A.P.$ 
        \item $A.P.-G.P.$ 
        \item $H.P.$
\end{enumerate}
\end{multicols} 

\item A tower stand at the centre of a circular park. $\vec{A}$ and $\vec{B}$ are two points on the boundary of the park such that AB $\brak{=a}$ subtends an angle of $60\degree$ at the foot of the tower, and the angle of elevation of the top of the tower from $\vec{A}$ or $\vec{B}$ is $30\degree.$ The height of the tower is \hfill\brak{2007}
\begin{multicols}{4}
\begin{enumerate}
        \item $\frac{a}{\sqrt{3}}$                    
        \item $a\sqrt{3}$ 
        \item $\frac{2a}{\sqrt{3}}$ 
        \item $2a\sqrt{3}$
\end{enumerate}
\end{multicols} 

\item AB is a vertical pole with $\vec{B}$ at the ground level and $\vec{A}$ at the top. A man finds that the angle of elevation the the point $\vec{A}$ from a certain point $\vec{C}$ on the ground is $60\degree.$ He moves away from the pole along the line BC to a point $\vec{D}$ such that $\text{CD}=7m.$ From $\vec{D}$ the angle of elevation of point $\vec{A}$ is $45\degree.$ Then the height of the pole is  

\hfill\brak{2008}
\begin{multicols}{4}
\begin{enumerate}
        \item $\frac{7\sqrt{3}}{2}\frac{1}{\sqrt{3}-1}m$          
        \item $\frac{7\sqrt{3}}{2}\brak{\sqrt{3}+1}m$ 
        \item $\frac{7\sqrt{3}}{2}\brak{\sqrt{3}-1}m$ 
        \item $\frac{7\sqrt{3}}{2}\frac{1}{\sqrt{3}+1}m$
\end{enumerate}
\end{multicols} 

\item For a regular polygon, let $r$ and $R$ be the radii of the inscribed and the circumscribed circles. A false statement among the following is \hfill\brak{2010}
\begin{enumerate}
        \item There is a regular polygon with $\frac{r}{R}=\frac{1}{\sqrt{2}}$                    
        \item There is a regular polygon with $\frac{r}{R}=\frac{2}{3}$ 
        \item There is a regular polygon with $\frac{r}{R}=\frac{\sqrt{3}}{2}$ 
        \item There is a regular polygon with $\frac{r}{R}=\frac{1}{2}$
\end{enumerate}

\item A bird is sitting on the top of a vertical pole $20m$ high and its elevation from a point $\vec{O}$ on the  ground is $45\degree.$ It flies off horizontally straight away from the point $\vec{O}$. After one second, the elevation of the bird from $\vec{O}$ is reduced to $30\degree.$ Then the speed in \brak{in\:m/s} of the bird is \hfill\brak{JEE M 2014}
\begin{multicols}{4}
\begin{enumerate}
        \item $20\sqrt{2}$                    
        \item $20\brak{\sqrt{3}-1}$ 
        \item $40\brak{\sqrt{2}-1}$ 
        \item $40\brak{\sqrt{3}-\sqrt{2}}$
\end{enumerate}
\end{multicols} 

\item If the angle of elevation of the top of a tower from three colinear points $\vec{A},\vec{B}\text{ and }\vec{C}$ on a line leading to foot of the tower, are $30\degree,45\degree \text{and}  60\degree$ respectively, then the ratio, $\text{AB}:\text{BC},$ is: 
\hfill\brak{JEE M 2015}
\begin{multicols}{4}
\begin{enumerate}
        \item $1:\sqrt{3}$                    
        \item $2:3$ 
        \item $\sqrt{3}:1$ 
        \item $\sqrt{3}:\sqrt{2}$
\end{enumerate}
\end{multicols} 

\item Let a vertical tower AB have its end $\vec{A}$ on the level ground. Let $\vec{C}$ be the  mid-point of AB and $\vec{P}$ be a point on the ground such that $\text{AP}=2\text{AB}.$ If $\angle{BPC}=\beta,$ then $\tan \beta$ is equal to: \hfill\brak{JEE M 2017}
\begin{multicols}{4}
\begin{enumerate}
        \item $\frac{4}{9}$                    
        \item $\frac{6}{7}$ 
        \item $\frac{1}{4}$ 
        \item $\frac{2}{9}$
\end{enumerate}
\end{multicols} 

\item $\Delta{PQR}$ is a triangular park with $\text{PQ}=\text{PR}=200m$. A T.V. tower stands at the mid-point of QR. If the angles of the elevation of the top of the tower at $\vec{P,Q \text{ and } R}$ are respectively $45\degree , 30\degree \text{ and } 30\degree,$ then the height of the tower \brak{in\:m} is: \hfill\brak{JEE M 2018}
\begin{multicols}{4}
\begin{enumerate}
        \item $50$                    
        \item $100\sqrt{3}$ 
        \item $50\sqrt{2}$ 
        \item $100$
\end{enumerate}
\end{multicols} 


\end{enumerate}
\end{document}
