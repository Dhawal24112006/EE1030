
%iffalse
\documentclass[journal]{IEEEtran}
\usepackage[a5paper, margin=10mm]{geometry}
%\usepackage{lmodern} % Ensure lmodern is loaded for pdflatex
\usepackage{tfrupee} % Include tfrupee package


\setlength{\headheight}{1cm} % Set the height of the header box
\setlength{\headsep}{0mm}     % Set the distance between the header box and the top of the text


%\usepackage[a5paper, top=10mm, bottom=10mm, left=10mm, right=10mm]{geometry}

%
\setlength{\intextsep}{10pt} % Space between text and floats

\makeindex


\usepackage{cite}
\usepackage{amsmath,amssymb,amsfonts,amsthm}
\usepackage{algorithmic}
\usepackage{graphicx}
\usepackage{textcomp}
\usepackage{xcolor}
\usepackage{txfonts}
\usepackage{listings}
\usepackage{enumitem}
\usepackage{mathtools}
\usepackage{gensymb}
\usepackage{comment}
\usepackage[breaklinks=true]{hyperref}
\usepackage{tkz-euclide} 
\usepackage{listings}
\usepackage{multicol}
\usepackage{xparse}
\usepackage{gvv}
%\def\inputGnumericTable{}                                 
\usepackage[latin1]{inputenc}                                
\usepackage{color}                                            
\usepackage{array}                                            
\usepackage{longtable}                                       
\usepackage{calc}                                             
\usepackage{multirow}                                         
\usepackage{hhline}                                           
\usepackage{ifthen}                                               
\usepackage{lscape}
\usepackage{tabularx}
\usepackage{array}
\usepackage{float}


\newtheorem{theorem}{Theorem}[section]
\newtheorem{problem}{Problem}
\newtheorem{proposition}{Proposition}[section]
\newtheorem{lemma}{Lemma}[section]
\newtheorem{corollary}[theorem]{Corollary}
\newtheorem{example}{Example}[section]
\newtheorem{definition}[problem]{Definition}
\newcommand{\BEQA}{\begin{eqnarray}}
\newcommand{\EEQA}{\end{eqnarray}}

\theoremstyle{remark}


\begin{document}
\bibliographystyle{IEEEtran}
\onecolumn

\title{JEE MAINS 28 Jun 2022 Shift-2}
\author{ee24btech11015 - Dhawal}
\maketitle

\renewcommand{\thefigure}{\theenumi}
\renewcommand{\thetable}{\theenumi}

\begin{enumerate}[start=11]
	\item Let the plane $ax + by + cz = d$ pass through $\myvec{2\\ 3\\ -5}$ and is perpendicular to the planes $2x + y - 5z = 10$ and $3x + 5y - 7z = 12.$ If $a, b, c, d$ are integers $d > 0$ and $gcd \brak{\abs{a}, \abs{b}, \abs{c}, d} = 1,$ then the value of $a + 7b + c + 20d$ is equal to :\hfill\brak{28/06/2022-Shift-2}

\begin{multicols}{4}
\begin{enumerate}
\item $18$
\item $20$
\item $24$
\item $22$
\end{enumerate}
\end{multicols}

\item  The probability that a randomly chosen one-one function from the set $\cbrak{a, b, c, d}$ to the set $\cbrak{1, 2, 3, 4, 5}$ satisfies $f\brak{a} + 2f\brak{b} - f\brak{c} = f\brak{d}$ is: \hfill\brak{28/06/2022-Shift-2}


\begin{multicols}{4}
\begin{enumerate}
\item $\frac{1}{24}$
\item $\frac{1}{40}$
\item $\frac{1}{30}$
\item $\frac{1}{20}$
\end{enumerate}
\end{multicols}

\item  The value of $ \lim_{n \to \infty} 6 \tan \cbrak{ \sum_{r=1}^{n} \tan^{-1} \cbrak{ \frac{1}{r^2 + 3r + 3} }}$
is equal to :\hfill\brak{28/06/2022-Shift-2}

\begin{multicols}{4}
\begin{enumerate}
\item $1$
\item $2$
\item $3$
\item $6$
\end{enumerate}
\end{multicols}

\item   
 $\vec{a}$ be a vector which is perpendicular to the \text{ vector } $3\hat{i} + \frac{1}{2}\hat{j} + 2\hat{k}.  \text{ If }  \vec{a} \times \brak{2\hat{i} + \hat{k}} = 2\hat{i} - 13\hat{j} - 4\hat{k}$, then the projection of the vector on the vector
$2\hat{i} + 2\hat{j} + \hat{k} $ is: 

\hfill\brak{28/06/2022-Shift-2}


\begin{multicols}{4}
\begin{enumerate}
\item $\frac{1}{3}$
\item $1$
\item $\frac{5}{3}$
\item $\frac{7}{3}$
\end{enumerate}
\end{multicols}

\item  If
$\cot{\alpha} = 1   \text{ and }   \sec{\beta} = -\frac{5}{3} \text{ where }   \pi < \alpha < \frac{3\pi}{2}   \text{ and }   \frac{\pi}{2} < \beta < \pi,$ then the value of $\tan \brak{\alpha + \beta}$ and the quadrant in which $\alpha + \beta$ lies, respectively are :

\hfill\brak{28/06/2022-Shift-2}


\begin{multicols}{2}
\begin{enumerate}
\item $\frac{-1}{7} $ and $4^{th}$ quadrant
\item $7 $ and $1^{st}$ quadrant
\item $-7$ and $4^{th}$ quadrant
\item $\frac{1}{7} $ and $1^{st}$ quadrant
\end{enumerate}
\end{multicols}

\end{enumerate}

\section*{B. Numericals}
\begin{enumerate}

\item  Let the image of the point $\vec{P}\myvec{1\\ 2\\ 3}$ in the line $L: \frac{x - 6}{3} = \frac{y - 1}{2} = \frac{z - 2}{3}$ be $\vec{Q}$. Let $\vec{R} \myvec{\alpha \\ \beta \\ \gamma}$ be a point that divides internally the line segment PQ in the ratio 1 : 3. Then the value of $22\brak{\alpha + \beta + \gamma}$ is equal to :\hfill\brak{28/06/2022-Shift-2}
\\


\item Suppose a class has 7 students. The average marks of these students in the mathematics examination is 62, and their variance is 20. A student fails in the examination if he/she gets less than 50 marks, then in worst case, the number of students can fail is:\hfill\brak{28/06/2022-Shift-2}
\\


\item If one of the diameters of the circle $x^2 + y^2 - 2\sqrt{2}x - 6\sqrt{2}y + 14 = 0$ is a chord of the circle
${\brak{ x - 2\sqrt{2} }}^2 + {\brak{ y - 2\sqrt{2} }}^2 = r^2$ , then the value of $r^2$ is equal to:

\hfill\brak{28/06/2022-Shift-2}
\\

\item If $\lim_{x \to 1} \frac{\sin\brak{3x^2 - 4x + 1} - x^2 + 1}{2x^3 - 7x^2 + ax + b} = -2$, then the value of $\brak{a - b}$ is equal to:

\hfill\brak{28/06/2022-Shift-2}
\\

\item Let for $n = 1, 2,\dots, 50, \; S_n$ be the sum of the infinite geometric progression whose first term is $n^2$ and whose common ratio is $\frac{1}{\brak{n + 1}^2}$. Then the value of $\frac{1}{26} + \sum_{n=1}^{50} \brak{ S_n + \frac{2}{n + 1} - n - 1} $ is equal to:\hfill\brak{28/06/2022-Shift-2}
\\


\item If the system of linear equations $2x - 3y = \gamma + 5, \alpha x + 5y = \beta + 1, \text{ where } \alpha, \beta, \gamma \in R$ has infinitely many solutions, then the value of $\abs{9\alpha + 3\beta + 5\gamma}$ is equal to:

\hfill\brak{28/06/2022-Shift-2}
\\

\item  Let $A = \myvec{1+\iota && 1\\-\iota &&0} $ where $\iota=\sqrt{-1}.$ Then, the number of elements in the set $ \cbrak{n \in \{1, 2, \ldots, 100\} : A_n = A} $  is:\hfill\brak{28/06/2022-Shift-2}\\

\item Sum of squares of modulus of all the complex numbers $z$ satisfying ${z} = \iota z^2 + z^2 - z$ is equal to:\hfill\brak{28/06/2022-Shift-2}
\\

\item Let $S = \cbrak{1, 2, 3, 4}.$ Then the number of elements in the set $\cbrak{f : S \times S \Longrightarrow S : f \text{ is onto and }f\brak{a,b}=f\brak{b,a} \geq a \; \forall \brak{a,b} \in  S \times S}$ is:

\hfill\brak{28/06/2022-Shift-2}
\\

\item  The maximum number of compound propositions, out of $p \lor r \lor s, \; p \lor r \lor \sim s, \; p \lor \sim q \lor s, \; \sim p \lor \sim r \lor s, \; \sim p \lor \sim r \lor \sim s, \; \sim p \lor q \lor \sim s, \; q \lor r \lor \sim s, \; q \lor \sim r \lor \sim s, \; \sim p \lor \sim q \lor \sim s$ that can be made simultaneously true by an assignment of the truth values to $p, q, r \text{ and } s,$ is equal to:\hfill\brak{28/06/2022-Shift-2}



\end{enumerate}


\end{document}

